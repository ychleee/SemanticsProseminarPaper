Related to

{[}{[}Semantics Proseminar Term Paper{]}{]} \# Abstract In this paper,
we propose a new account of the so-called evidential \emph{-te-} in
Korean assertions. Our perspective differs from previous studies by Lee
(2010, 2013) and Lim (2010), who suggest that \emph{-te-} is an
evidential marker. Instead, we align with Chung's (2007) argument that
\emph{-te-} is a (spatio-)temporal deictic tense that signifies the
perspective on which the assertion is based. However, we diverge from
Chung's view by denying that (tense) morphemes such as \emph{-ass-},
\emph{-keyss-} and ø-morpheme are evidential markers in \emph{-te-}
sentences. C:: Hee: What does it mean to ``signify''? If \emph{-te-} is
a tense marker, can it ``signify perspectives''?

We argue that in Korean \emph{-te-} assertive sentences, there is no
evidential morpheme that marks {[}{[}The Deictic Core of
`Non‐Experienced Past' in Cuzco
Quechua\#\^{}71d4a3\textbar propositional-level
evidentiality{]}{]}\footnote{Faller (2004), ``The Deictic Core of
  `Non‐Experienced Past' in Cuzco Quechua''}, which is a relation
between the speaker and the proposition. Instead, we propose that the
propositional-level evidentiality of \emph{-te-} assertions comes from
conversational implicature of plain assertions. In our account,
\emph{-te-} assertions are understood as plain assertions whose
evidential vantage point is marked. By virtue of being a version of
plain assertion, it already has a propositional-level evidentiality of
assertions. Therefore, \emph{-te-} sentences do not require an
additional marker for propositional-level evidentiality. The role of
\emph{-te-} is to specify the vantage point on which the assertion's
evidence is based.

We will demonstrate that our account can comprehensively explain the
data presented in Lee, Lim, and Chung's studies and does not encounter
the problems each of their accounts faced.

Ingredients

{[}{[}Supplementary Material for Semantics Term Paper\#Abstract
Ingredients\textbar Abstract Ingredients{]}{]} \# 1. Introduction
\#\#\#\#\#\# ingredients: first sentence - Chung 2007: The status of the
Korean suffix -te presents an intriguing puzzle and is the source of
much controversy in the Korean literature. - Lee 2013: This paper
explores how the meaning of evidentiality is temporally constrained in
nat- ural language, through a detailed study of the meaning of Korean
evidential sentences. - Lim 2012: Korean verbal ending -te- introduces
different presuppositions depending on whether it appears with a tense
marker (such as the past tense marker -ass/ess- or the future tense
marker -keyss-) or not. \#\# 1.1. Evidentiality and Korean \emph{-te-}
Korean verbal ending \emph{-te-} has been analyzed as contributing to
the evidentiality of a given assertion in recent formal theoretic
studies (cf.~{[}{[}Spatial deictic tense and evidentials in
Korean\textbar Chung, 2007{]}{]}; {[}{[}Temporal constraints on the
meaning of evidentiality\textbar Lee, 2013{]}{]}; {[}{[}Korean
Evidential -te- Inference from Direct Evidence\textbar Lim, 2012{]}{]}).
In these papers, it has been commonly argued that \emph{-te-} marks that
the speaker's direct perception or sensory observation of the evidence
for the asserted proposition took place at some past reference time.
Following are the common and prototypical examples of the papers
mentioned above (directly quoted from Lee, 2013):

\ex. Context: Yenghi saw it raining yesterday. Now, she says: \gll Pi-ka
o-\(\emptyset\)-te-la.\textbackslash{} rain-NOM
fall-PRES-TE-DEC\textbackslash{} `{[}I made a sensory observation
that{]} it was raining.'

Different accounts diverged on how to explain the fact that Korean
\emph{-te-} sentences can in principle offer direct or indirect evidence
for the proposition asserted. Interactions with tense(-aspect)
morphemes\footnote{Although these bold-typed morphemes are often
  analyzed as PST or FUT, some kind of temporal morpheme. We will argue
  that \emph{-ass-} is a perfective aspectual marker and \emph{-keyss-}
  an epistemic modal. These non-tense markers will properly interact
  with the Evidential Deictic Tense that \emph{-te-} denotes, which is
  our core assumption.} seemed to decide on which evidence type was
available to the speaker.

\ex. Context: Yenghi saw yesterday that the ground was wet. Now, she
says: \gll Pi-ka o-\textbf{ass}-te-la.\textbackslash{} rain-NOM
fall-\textbf{PST}-TE-DEC\textbackslash{} `{[}I inferred (from the
acquired sensory evidence) that{]} it had rained.'

\ex. Context: Yenghi saw the overcast sky yesterday. Now, she says:
\gll Pi-ka o-\textbf{keyss}-te-la.\textbackslash{} rain-NOM
fall-\textbf{FUT}-TE-DEC\textbackslash{} `{[}I inferred (from the
acquired sensory evidence) that{]} it would rain.'

As one can see in (2)-(3), presence of some tense-related morphemes give
rise to an inferential interpretation of the \emph{-te-} sentences. In
contrast, a zero-morpheme in (1), which is assumed to be a present
morpheme in Korean, lets the speaker assert the proproposition based on
a direct perception evidence.

From this spectrum of data, Korean \emph{-te-} has often been argued to
be an evidential marker (but see Chung, 2007). In the present paper, we
will argue that this morpheme is not an evidential marker, at least in
the sense that it does not mark any evidence at a propositional level.
Korean \emph{-te-} is a marker of perspective, which specifies the
vantage point on which the assertion's evidence is based. It will be
argued that an assertion's evidence is required independently of
evidentials since (evidential-less) plain assertions in Korean cannot be
uttered felicitously without strong/appropriate-type-of evidence. As
with {[}{[}Spatial deictic tense and evidentials in Korean\textbar Chung
(2007){]}{]}, \emph{-te-} will be treated as a \emph{Evidential Deictic
Tense} (similar to \emph{Evidence Acquisition Time} in previous
literature on evidentiality).

In the remainder of this introduction, a brief summary of constraints on
\emph{-te-} sentences will be offered. Before we could propel this
rather unexpected argument, we present an overview of three main
previous accounts along with problems each causes in
{[}{[}MVP-Evidentiality as Perspectiveness- Analysis of Korean morpheme
`-te-'\#2. Overview of previous accounts\textbar Section 2{]}{]}. In
{[}{[}MVP-Evidentiality as Perspectiveness- Analysis of Korean morpheme
`-te-'\#3. Our Account: `더' as marking perspective\textbar Section
3{]}{]}, we provide our account, first by clarifying what \emph{-te-} is
not and then showing that while \emph{-te-} is not an evidential, there
is evidentiality in assertions in general regardless of evidential
markers in Korean. In {[}{[}MVP-Evidentiality as Perspectiveness-
Analysis of Korean morpheme `-te-'\#4. Comparisons with previous
accounts\textbar Section 4{]}{]}, a comparison of our account with
previous ones will hopefully show that ours is on the right track. Then,
in {[}{[}MVP-Evidentiality as Perspectiveness- Analysis of Korean
morpheme `-te-'\#5. Cross-linguistic implications\textbar Section
5{]}{]}, a cross-linguistic implication of our argument will be
reviewed. {[}{[}MVP-Evidentiality as Perspectiveness- Analysis of Korean
morpheme `-te-'\#6. Conclusion\textbar Section 6{]}{]} concludes.

\hypertarget{constraints-on--te-}{%
\subsection{\texorpdfstring{1.2. Constraints on
\emph{-te-}}{1.2. Constraints on -te-}}\label{constraints-on--te-}}

Following constraints are commonly assumed in the papers under
discussion. We present a summary of each constraint, with the glossing
slightly modified for a unified demonstration.

\hypertarget{speakers-personal-observation-constraint}{%
\subsubsection{1.2.1. Speaker's Personal Observation
Constraint}\label{speakers-personal-observation-constraint}}

In paradigmatic cases, a sentence with \emph{-te-} must describe a
situation that the speaker witnessed (Chung, 2007; a.o.). When there is
no perfective aspect (\emph{-ass/ess-}\footnote{These two morphemes are
  allomorphs. Decision among these two are purely phonological.}) in a
\emph{-te-} sentence, sentence (5) is an infelicitous utterance from a
speaker who is not a contemporary of Shakespeare. (Chung further
supports this constraint by noting that the context where a speaker saw
a movie of Shakespeare's biography would save 5. We agree with this
observation.)

\%\%4\%\% \ex. \gll. ku tangsi mary-ka ce cip-ey
sal-te-la.\textbackslash{} that time Mary-NOM that house-LOC
live-TE-DEC\textbackslash{} `{[}I saw{]} Mary was living in that house
at that time.'

\%\%5\%\% \ex. \gll *ku tangsi shakespeare-ka ce cip-ey
sal-te-la.\textbackslash{} that time Shakespeare-NOM that house-LOC
live-TE-DEC\textbackslash{} (Intended meaning) `{[}I saw{]} Shakespeare
was living in that house at that time.'

\hypertarget{equi-subject-constraint}{%
\subsubsection{1.2.2. Equi-Subject
Constraint}\label{equi-subject-constraint}}

Chung (2007) further explains the well-known (among Korean linguists)
constraints on Korean \emph{-te-} sentences: In case of sensory or psych
predicates such as \emph{oylop-} (be lonely), \emph{coh-} (like), the
subject of a \emph{-te-} sentence must be the speaker.\footnote{Why?
  because, in {[}{[}Anscombe, G. E. M.\textbar anscombe{]}{]}'s sense,
  it is not that we know our actions from our observation. We can make
  it more precise: in case actions, we cannot use \emph{-te-}, because
  it is, in some sense, unobservable by us.}

\%\%6\%\% \ex. \gll ku-ttay(-nun) nay-ka oylop-te-la.\textbackslash{}
that-time(-TOP) I-NOM be.lonely-TE-DEC\textbackslash{} `{[}I felt{]} I
was lonely at that time.'

\%\%7\%\% \ex. \gll *ku-ttay(-nun) mary-ka oylop-te-la.\textbackslash{}
that-time(-TOP) Mary-NOM be.lonely-TE-DEC\textbackslash{} (Intended
meaning) `{[}I felt{]} Mary was lonely at that time.'

\%\%8\%\% \ex. ku-ttay(-nun) nay-ka mary-ka coh-te-la.\textbackslash{}
that-time-TOP I-NOM Mary-NOM like-TE-DEC\textbackslash{} `{[}I felt{]} I
liked Mary at that time.'

\%\%9\%\% \ex. \gll *ku-ttay(-nun) john-i mary-ka
coh-te-la.\textbackslash{} that-time-TOP John-NOM Mary-NOM like-TE-DEC
(Intended meaning) `{[}I felt{]} John liked Mary at that time.'

An additional restriction is that not all sensory or emotional
predicates but only unaccusative ones whose subject is an experiencer
(and not an agent) follows this constraint. Note that the auxiliary
\emph{-eha-}(do) verb in the following sentences changes the
acceptability of 1st person subject - sensory/psych predicate
association in \emph{-te-} sentences. Sentences (10)-(11) are not
completely ungrammatical less infelicitous, but they blatantly sound
unnatural in that the speaker is reporting him- or herself's (emotional)
actions as if he or she were observing someone else's.

\%\%10\%\% \ex. \gll ??ku-ttay-nun nay-ka mopsi
oylow-\textbf{eha}-te-la.\textbackslash{} that-time-TOP I-NOM awfully
be.lonely-do-TE-DEC\textbackslash{} (Intended meaning) `{[}I felt{]} I
was terribly lonely at that time.'

\%\%11\%\% \ex. \gll ??ku-ttay-nun nay-ka mary-lul
coh-\textbf{aha}-te-la.\textbackslash{} that-time-TOP I-NOM Mary-ACC
like-do-TE-DEC\textbackslash{} (Intended meaning) `{[}I felt{]} I liked
Mary at that time.'

\hypertarget{non-equi-subject-constraint}{%
\subsubsection{1.2.3. Non-Equi-Subject
Constraint}\label{non-equi-subject-constraint}}

For all predicates except for sensory and psych predicates, the subject
of sentence cannot be the speaker. Predicates such as \emph{ka-} (go) or
\emph{yeyppu-} (be.pretty) do not go well with 1st person subject in
\emph{-te-} sentences. The unnaturalness of the \emph{-te-} sentences
with 1st person subject is the same as sentences (10)-(11).

\%\%12\%\% \ex. \gll mary-ka/nun hakkyo-ey ka-te-la.\textbackslash{}
Mary-NOM/TOP school-LOC go-TE-DEC\textbackslash{} `{[}I saw{]} Mary was
going to school.'

\%\%13\%\% \ex. \gll ??nay-ka/na-nun hakkyo-ey ka-te-la.\textbackslash{}
I-NOM/I-TOP school-LOC go-TE-DEC\textbackslash{} (Intended meaning)
`{[}I saw{]} I was going to school.'

\%\%14\%\% \ex. \gll mary-ka/nun yeyppu-te-la.\textbackslash{}
Mary-NOM/TOP be.pretty-TE-DEC\textbackslash{} `{[}I saw{]} Mary was
pretty.'

\%\%15\%\% \ex. \gll ??nay-ka/na-nun yeyppu-te-la.\textbackslash{}
I-NOM/I-TOP be.pretty-TE-DEC\textbackslash{} (Intended meaning) `{[}I
saw{]} I was pretty.'

\hypertarget{te--always-requires-direct-perceptual-evidence.}{%
\subsubsection{\texorpdfstring{1.2.4. \emph{-te-} always requires direct
perceptual
evidence.}{1.2.4. -te- always requires direct perceptual evidence.}}\label{te--always-requires-direct-perceptual-evidence.}}

Lim (2012), referring to {[}{[}Lee (2010){]}{]}, point sout that usage
of \emph{-te-} in assertions requires the speaker's direct perception of
some eventuality. Take a look at the following example:

\%\%16\%\% \ex. (Scenario: John is sick, so he has stayed in his bed
since yesterday and has not been outside at all. John's room does not
have any window, so he could not see outside, either. \emph{Today he
heard from his roommate that the ground is wet.} John says to his friend
on the phone\ldots) \gll \#Ecey pi-ka o-ass-te-la\textbackslash{}
Yesterday rain-NOM fall-PST-te-DECL\textbackslash{} `{[}I inferred (from
the acquired sensory evidence) that{]} Yesterday it rained'

However, even in a similar scenario, when the speaker does have some
kind of direct perception of the eventuality under consideration, it
becomes felicitous to utter a \emph{-te-} utterance:

\%\%17\%\% \ex. (Scenario: Mary was sick, so she stayed in her bed all
day long. Her room does not have any window, so she could not see
outside at all. \emph{Now she saw that her roommate came home with a wet
umbrella and a wet raincoat.} She says to her friend on the phone\ldots)
\gll Onul pakk-ey pi-ka o-te-la.\textbackslash{} Today outside-LOC
rain-NOM fall-te-DECL\textbackslash{} (Intended meaning) `{[}I made a
sensory observation that{]} Today it rained outside'

More Thinking

\begin{itemize}
\tightlist
\item
  이게 정확히 무슨 Constraint? Evidence에 대한 Herasay같은것을
  제외하려고 하는 것인가? =\textgreater{} Here, `direct perception of
  evidence' can be misleading. In some sense, we can say that John had
  directly perceived his evidence, say, his roomates saying that the
  ground is wet. Why John's case is infellicitous? Because `hearsay' is
  not counted as direct perceptive evidence. What if someone says, it
  involves direct perception of the evidence because it involves direct
  perception of John's friends' saying? It is a bit wrong because here,
  `direct'
\end{itemize}

\begin{quote}
{[}!warning{]} This is misleading \textgreater{} Acquiring evidence
almost always requires direct perception of some sort. Hearsay requires
direct perception of sounds. Listening to news requires direct
perception of sounds. \textgreater{} So, this says almost nothing. No
substantial constraint at all. \textgreater{} What this case shows is
\end{quote}

\begin{itemize}
\tightlist
\item
  But is it?

  \begin{itemize}
  \tightlist
  \item
    어제 오바마가 왔더라.
  \item
    ``Listening to the news in the morning,''
  \end{itemize}
\item
  그러면 뉴스 들은거는? 뉴스 들은거는 direct perception of evidnence
  인가? 그것과 Hearsay의 차이는?
\end{itemize}

1.2.5. More data(Lim ) \emph{-te-} may introduce inferential evidential
presupposition without any overt tense. {[}{[}Korean Evidential -te-
Inference from Direct Evidence\#2.2 -te- may introduce inferential
evidential presupposition without any overt tense.\textbar Lim
(2012){]}{]}

\begin{itemize}
\tightlist
\item
  오늘 밖에 비가 오더라.

  \begin{itemize}
  \tightlist
  \item
    맥락: (Scenario: Mary was sick, so she stayed in her bed all day
    long. Her room does not have any window, so she could not see
    outside at all. Now she saw that her roommate came home with a wet
    umbrella and a wet raincoat. She says to her friend on the
    phone\ldots)\\
    C:: 정에 대한 반박?
  \end{itemize}
\end{itemize}

1.2.6. More data(Lim) The prejacent of \emph{-te-} may denote future
eventuality without any overt tense. {[}{[}Korean Evidential -te-
Inference from Direct Evidence\#2.3 The prejacent of -te- may denote
future eventuality without any overt tense\textbar Lim (2012){]}{]}

존이 WCCFL에서 논문을 발표하더라 - John presented a paper in WCCFL -
John is going to present a paper in WCCFL (이 경우엔 '발표하-겠-더라'도
가능) \# 2. Overview of previous accounts \#\# 2.1. {[}{[}Spatial
deictic tense and evidentials in Korean\textbar Chung (2007){]}{]} As
the first formal explanation of the so-called Korean evidential morpheme
\emph{-te-}, {[}{[}Chung (2007){]}{]} argues that \emph{-te-} is not an
evidential morpheme. Instead, the tense morphemes that immediately
precede \emph{-te-}, such as \emph{-ass/ess-} (PST), \emph{-keyss-}
(FUT) or even a zero-morpheme (PRS) are evidentials in a \emph{-te-}
sentence. \emph{-te-} is a spatio-temporal deictic (past) tense in
Faller's 2004 sense.

{[}{[}The Deictic Core of `Non‐Experienced Past' in Cuzco
Quechua\textbar Faller (2004){]}{]} argues that \emph{-sqa} in Cuzco
Quechua is a non-experienced past marker, that requires that the
eventuality of the prejacent took place outside of the speaker's
perceptual field (the time-location coordinate of the speaker's
perception). It attaches to verbs, differently from Cuzco Quechua
evidentials \emph{-mi}, \emph{-chá}, \emph{-si} (cf.~{[}{[}Faller
(2002){]}{]}) which attach to other constituents, hence it cannot be an
evidential but a tense. Adopting this way of thinking, Chung (2007)
argues that \emph{-te-} in Korean is equally a spatio-temporal deictic
tense, minimally differing in that \emph{-te-} requires that the
eventuality under discussion (somehow) took place \emph{inside} the
speaker's perceptual field.

According to Chung (2007, Sect. 3.4), simple deictic tense morphemes
\emph{-ass/ess-}, \emph{-keyss-} or \(\emptyset\) are actually ambiguous
between tense and evidentials, the latter of which is only available
when the spatio-temporal deictic past tense is marked with
\emph{-te-}\footnote{Chung (2007) further shows that this special type
  of deictic tense has its present counterpart: \emph{-ney}. Although we
  do not agree with Chung's analysis of whether these morphemes are
  evidentials or not, our account do align with this temporally parallel
  analysis of \emph{-te-} and \emph{-ney}.} As evidentials, the overt
forms \emph{-ass/ess-} and \emph{-keyss} serve as indirect evidentials
(examples 2 and 3), and the covert form (\(\emptyset\)) serves as a
direct evidential (example 1).

Chung's account covers most of the data that we presented in
Introduction: direct vs.~indirect distinction in prototypical cases (1
vs.~2,3); the (Non-)Equi-Subject Constraints. For non-observable
predicates such as \emph{oylop-} (be.lonely), \emph{coh-} (like), the
speaker's perceptual field cannot be at someone else's emotional state
(hence the Equi-Subject Constraint). In case of observable predicates
(and transitive psych predicates with \emph{-eha}), it is plainly
redundant to be reporting the observation on oneself resulting in the
unnaturalness in examples (10,11,13,15), hence the Non-Equi-Subject
Constraint.

However, there are both conceptual and empirical problems to Chung's
analysis, both of which are nonnegligible. First, Chung's classification
of evidentials cannot explain the data in (17). For the proposition
`Today it rained outside', Mary does not have any direct evidence. She
only has an indirect (albeit reliable) evidence for raining outside
today, which is her perception of a wet umbrella and a wet raincoat.
Still, we do not see any (simple tense) morphemes \emph{-ass/ess-} or
\emph{-keyss-}, which Chung argue to be indirect evidentials. The
relation to the evidence is not predicted based on the presence/absence
of these morphemes.

Moreover, as Lee (2013:33-34) criticizes, Chung's ambiguity analysis is
only motivated by her typological assumption:

\ex. (\ldots) --te itself is not an evidential. The very purpose of an
evidential system is to distinguish direct and indirect evidence, and
thus it is unlikely that both direct evidence and indirect evidence are
expressed by the same morpheme. (Chung 2007, p.~195)

Cross-linguistically, in one of the widespread two-fold evidential
system ({[}{[}Aikhenvald (2004){]}{]}, A3-system), ``the distinction
between direct evidence vs.~inferential evidence is not marked by
seperate morphemes'\,' (Lee, 2013:34). Evidentials may code the source
of informations, but the (in)directness of the evidence do not have to
be encoded lexically in separate morphemes. The example (17) seems to
suggest that this is the case in Korean.

\hypertarget{temporal-constraints-on-the-meaning-of-evidentialitylee-2010-2013}{%
\subsection{2.2. {[}{[}Temporal constraints on the meaning of
evidentiality\textbar Lee (2010,
2013){]}{]}}\label{temporal-constraints-on-the-meaning-of-evidentialitylee-2010-2013}}

\hypertarget{account}{%
\subsubsection{account}\label{account}}

\begin{enumerate}
\def\labelenumi{\arabic{enumi}.}
\tightlist
\item
  `더' marks sensory evidence.

  \begin{enumerate}
  \def\labelenumii{\arabic{enumii}.}
  \tightlist
  \item
    there are lingering problems:
  \end{enumerate}
\item
  `더' expresses necessary modal \#\#\# critique \#\#\#\# 1. `더'
  expressing necessary epistemic modal?\\
\end{enumerate}

\begin{enumerate}
\def\labelenumi{\arabic{enumi})}
\tightlist
\item
  비가 오고 있었다.
\item
  비가 오더라.
\item
  비가 오고 있던 게 분명해./비가 오고 있었다고 확신해/분명히 생각해. (as
  epistemic modal) // (나는 비오는 걸 본 게) 확실해. (meta-discourse)
\end{enumerate}

\begin{itemize}
\tightlist
\item
  3)보다 2)가 강하다. 3)은 직접 보지 못했다는 뉘앙스가 있음. It cannot
  be equated with epistemic modal. \#\#\#\# 2. `더' sentence isn't
  restricted to sensory cases.
\item
  그때 욕심이 나더라.
\item
  그때 생각 나더라. \#\#\#\# 3. The meanign of `더' cannot be reduced to
  certainty.
\end{itemize}

\begin{enumerate}
\def\labelenumi{\arabic{enumi}.}
\tightlist
\item
  그때 내가 좋은 생각이 났다. (fellicitous)
\item
  그때 내가 계획을 짜고 있더라. (infellicitous)
\end{enumerate}

\begin{itemize}
\item
  1, 2 has no difference in terms of certainty. They are both certain to
  the speaker. But 1 is acceptable while 2 is not. We need some other
  criterion.
\item
  evidentiality of `더' is not reducible to degree of certainty (contra
  Lee)
\item
  -te- is a direct evidential marker. Its indirect meaning comes from
  the temporal constraints \#\#\#\# Clarification: Why Lee's
  understanding of `sensory perception' is problematic:
\item
  뉴스 보고 '오바마 오더라'라고 말하는 경우

  \begin{itemize}
  \tightlist
  \item
    Is this the case of sensory perception?
  \item
    Surely, it involves senseory perception in the process, because we
    actually watched \#\# 2.3. {[}{[}Korean Evidential -te- Inference
    from Direct Evidence\textbar Lim (2012){]}{]} \#\#\# account
  \end{itemize}
\end{itemize}

\hypertarget{critique}{%
\subsubsection{critique}\label{critique}}

\hypertarget{comparsion-and-assessment}{%
\subsection{Comparsion and assessment}\label{comparsion-and-assessment}}

\begin{itemize}
\tightlist
\item
  Desiderata for `-te' sentences
\end{itemize}

\hypertarget{our-account}{%
\subsection{Our account}\label{our-account}}

\begin{quote}
{[}!note{]} Our account of -te- 1. -te- is not an evidential morpheme
that marks propositional-level evidentiality 2. -te- assertive sentence
is an assertive sentence that asserts an event P is the case. 3. `-te-'
is a deictic tense that marks the vintage point the event is asserted
\end{quote}

We argue that in Korean -te- assertive sentences, there is no evidential
morpheme that marks {[}{[}The Deictic Core of `Non‐Experienced Past' in
Cuzco Quechua\#\^{}71d4a3\textbar propositional-level
evidentiality{]}{]}\footnote{Faller (2004), ``The Deictic Core of
  `Non‐Experienced Past' in Cuzco Quechua''}, which is a relation
between speaker and the proposition. Instead, we propose that the
propositional-level evidentiality of -te- assertions comes from
conversational implicature of plain assertions. In our account, -te-
assertions are understood as plain assertions whose evidential vantage
point is marked. As it is already a version of plain assertion, it
already has propositional-level evidentiality of assertive sentences.
Therefore, -te- sentences do not need an additional marker for
propositional-level evidentiality. The role of ``-te-'' is to specify
the vantage point on which the assertion's evidence is based on.

\hypertarget{our-account-uxb354-as-marking-perspective}{%
\section{3. Our Account: `더' as marking
perspective}\label{our-account-uxb354-as-marking-perspective}}

\hypertarget{thesis-1-what--te--is-not}{%
\subsection{\texorpdfstring{Thesis (1) what \emph{-te-} is
not}{Thesis (1) what -te- is not}}\label{thesis-1-what--te--is-not}}

\hypertarget{te--is-not-necessary-modal.-contra-lee}{%
\subsubsection{-te- is not necessary modal. (Contra
Lee)}\label{te--is-not-necessary-modal.-contra-lee}}

We propose a new understanding of the Korean ``-te-'' marker in
assertive sentences. Our perspective diverges from previous studies,
particularly Lee (2013) who analyzed -te- as a necessity modal. However,
we argue that -te- is not a necessity modal.

Lee (2013) presents a lexical entry for -te- as seen in data 22, which
is as follows:
\([[-te]] = \lambda P_{<s,<i,t>>}.\lambda w.\lambda t. \exists t''[t''<t \land \forall w'[w' \in BEST(SO,ST/DX, w, t'') \rightarrow P(w')(t''))]]\)
This lexical entry, however, faces difficulty in explaining data such as
the following scenario:

Context1: There was a murder in the neighborhood of John: Bill was
murdered by someone. John is a close friend to Bill. He lives next door
and he is one of the officials who investigated the incident. John saw
that Bill and Tom fought heavily the day before the murder. (Tom said
``I will kill you.'') And he saw that Tom buying a gun and bullets right
after the fight. What he also saw was Tom going toward the direction to
Bill's house on the day of the murder. And by investigating the
situation, he found out that the type of bullet that Tom bought the day
before the murder was the same kind of bullet that Bill had been shot.
The next day of the murder, John talks to his officer friend about the
murder. 1a. \# ``톰이 빌을 쏜 것이 틀림없어'' (It must be the case that
Tom shoot Bill.) 1b. ``톰이 빌을 쐈더라.'' ({[}It is observed by the
speaker{]} Tom shoot Bill.)

\begin{enumerate}
\def\labelenumi{\arabic{enumi}.}
\tightlist
\item
  In this situation, John can felicitously assert the necessity
  epistemic modal: ``톰이 빌을 쏜 것이 틀림없어'' (It must be the case
  that Tom shoot Bill.)
\item
  But, John cannot felicitously assert -te- sentence: ``톰이 빌을
  쐈더라.'' ({[}It is observed by the speaker{]} Tom shoot Bill.)
\item
  But by if we follow Lee (2013)'s analysis of \emph{-te-} and take
  lexical entry of \([[-te-]]\), John can felicitously assert that
  ``톰이 빌을 쐈더라.''
\end{enumerate}

1 is because given his sensory observations of evidences (modal base)
and stereotypical/doxastic ordering source, the proposition that Tom
shoot Bill is true in all epistemic alternative worlds for John.
However, it is infellicitous speak assert -te- sentence ``톰이 빌을
쐈더라.''

This example illustrates that Lee's lexical entry for -te- is unable to
explain why John cannot felicitously assert the sentence ``톰이 빌을
쐈더라'' using -te-, even though the evidence he acquired through
sensory observation is strong enough to eliminate epistemic
(stereotypical/doxastic) alternatives. In contrast, John can
felicitously assert the sentence in different situations in which he
watched the murder scene himself.

Context2: John saw the scene Tom shooting Bill. 2a. \# ``톰이 빌을 쏜
것이 틀림없어'' (It must be the case that Tom shoot Bill.) 2b. ``톰이
빌을 쐈더라.'' ({[}It is observed by the speaker{]} Tom shoot Bill.)

This suggests that in order to make -te- assertion, John is not just
required to have sensory observation of evidence, but is required to
directly observation of the eventuality. In other words, not only
sensory observation that has strong modal connection to the event is
required for John, but sensory observation of the event itself is
required to assert using `-te-' in this context. Lee's lexical entry
does not have that kind of constraints (sensory observation of the event
itself) so it wrongly predicts that it is felicitous to speak 1b.

Draft and ingredient

{[}{[}Supplementary Material for Semantics Term Paper\#Thesis 1: What
-te- is not{]}{]}

\hypertarget{te--is-not-a-morpheme-that-marks-propositional-level-evidentiality.}{%
\subsubsection{- te- is not a morpheme that marks propositional-level
evidentiality.}\label{te--is-not-a-morpheme-that-marks-propositional-level-evidentiality.}}

We argue that -te- is not an evidential marker. Our perspective diverges
from the mainstream views in previous studies, notably Lee (2013) and
Lim (2010) who argue that -te- is a morpheme that marks evidentiality.

In this respect, our account is similar to Chung (2007)'s account. Chung
also argues that -te- is not an evidential marker. Instead, Chung
proposed that morphemes like -ass- or -kyess-, or ø, when appeared in
-te- sentences, play the role of evidential marker that contributes the
meaning of evidentiality to -te- sentences. Morphemes like -ass- or
-kyess- are tense morphemes in other contexts, but with -te-, they
become evidentials. This ambiguity thesis of Korean tense morphemes,
however, has faced serious objections, as we have seen in previous
chapters. It is neither independently motivated nor correctly predicts
indirect/direct distinctions.

We acknowledge that there is a propositional-level evidentiality in -te-
assertive sentences, in that the speaker must have an appropriate level
of evidence in order to felicitously assert the -te- sentence. However,
we argue that this evidential element is not added by `-te-', as Lee and
Lim argued, or by `-ass-,' `kyess,' or ø, as in Chung's account.
Instead, we argue that this evidentiality is the result of
conversational implicature of assertion, not something that is
contributed by an evidential marker.

To show this, we will argue that plain assertive Korean sentences
without -te-, still exhibits the propositional level evidentiality. In
fact, in the same context, there is no difference in evidential
requirement for the speaker concerning evidence types with plain
assertions and -te- assertions. They both require the strongest type of
evidence: observational evidence.

Our argument that -te- is not an evidential marker is that it does not
contribute evidential meaning to a plain assertion. In most cases, if we
delete -te- from -te- assertion to get plain assertion, the evidential
requirement still remains. Let us consider the murder cases again.

context 1: (John did not observed the murder scene) John is saying to
his friend. 1a.''톰이 빌을 쏜 것이 틀림없어.'' It must be the case that
Tom shoot Bill. 1b. \#``톰이 빌을 쐈더라.'' {[}the speaker observed
that{]} Tom shoot Bill. 1c. \#``톰이 빌을 쐈어.'' Tom shoot Bill.

context 2: (John observed the murder scene) John is saying to his
friend, 2a. \# ``톰이 빌을 쏜 것이 틀림없어.'' 2b. ``톰이 빌을 쐈더라.''
{[}The speaker observed that{]} Tom shoot Bill 2c. ``톰이 빌을 쐈어.''
Tom shoot Bill

In the first context, John did not observe the murder scene himself, so
he cannot felicitously assert the -te- sentence ``톰이 빌을 쐈더라.'' or
the plain assertion ``톰이 빌을 쐈어.'' as he does not have direct
observational evidence. He can only felicitously assert the necessity
epistemic modal ``톰이 빌을 쏜 것이 틀림없어.'' which denotes that given
his sensory observations of evidences and stereotypical/doxastic
ordering source, the proposition that Tom shoot Bill is true in all
epistemic alternative worlds for John.

On the other hand, in the second context, John has directly observed the
murder scene, so he can felicitously assert the -te- sentence ``톰이
빌을 쐈더라.'' and the plain assertion ``톰이 빌을 쐈어.'' as he has
direct observational evidence.

This example illustrates that -te- is not an evidential marker as it
does not contribute evidential meaning to a plain assertion. The
evidential requirement remains the same in both plain assertions and
-te- assertions, which requires the strongest type of evidence:
observational evidence. This shows that the evidential requirement is
not added by -te-, contra Lee (2013), and Lim (2011). Additionally, it
also refutes the ambiguity thesis of Korean tense morphemes proposed by
Chung (2007) as -te- does not affect the evidential requirement of
assertions with morphemes -kyess-, `ess', or ø. Chung's account predicts
that sentences with -kyess-, `ess', or ø without -te- would not exhibit
evidentiality. But this is not so, because as we can see in 2c, plain
assertion also exhibits same type of evidential requirement for the
speaker as 2b, -te- assertion.

In summary, our argument is that -te- is not a morpheme that marks
propositional-level evidentiality. Instead, we proposed that the
propositional-level evidentiality in -te- assertive sentences is the
result of conversational implicature of assertion, not something that is
contributed by an evidential marker. Our argument is supported by the
fact that plain assertive Korean sentences without -te- still exhibit
the propositional level evidentiality and require the same level of
evidence. \#\#\#\#\#\# Ingredients {[}{[}Supplementary Material for
Semantics Term Paper\#Thesis 1: -te- is not an evidential marker{]}{]}

\hypertarget{thesis-2--te--as-marking-perspectivevantage-point--te-}{%
\subsection{\texorpdfstring{Thesis (2) \emph{-te-} as marking
perspective(vantage point)
-te-}{Thesis (2) -te- as marking perspective(vantage point) -te-}}\label{thesis-2--te--as-marking-perspectivevantage-point--te-}}

Then what does -te- contribute? We propose that -te- is a deictic tense
that denotes an EAT (Evidence Acquisition Time) that is prior to ST
(Speech Time) and that this EAT is an evidential perspective from which
the speaker meets this evidential requirement. Informally speaking,
`-te-' locates the assertion's evidential vantage point at a time before
ST. And from that vantage point, the speaker is required to meet the
evidential requirement that is created by conversational implicature of
assertion in that context. To illustrate this, let us start from plain
assertive sentence without -te-.

3a. 지금 밖에 폭풍쳐. It is storming outside now

Griciean Maxim of Quality of conversational implicature says that one
should assert things only one has appropriate evidence. We will add up
that rule a bit and argue that conversational implicature requires this
rule: An assertion's strength should match the evidence's strength. (The
assertion's strength must be proportional to one's evidence's strength.
) By this Implicature, plain assertions requires the strongest type of
evidence because they are strongest type of assertion. In most contexts,
this means that plain assertion requires a direct evidence, which is the
strongest type of evidence. So the sentence 3a, which is a plain
assertion, requires a direct observational evidence, or something on the
par, for a speaker to felicitously say it. This evidential requirement
for speaker doesn't change if we add `-te-' to the sentence.

3b. 지금 밖에 폭풍 치더라. {[}The speaker observed that{]} It is raining
outside now.

In case of -te- sentences like 3b), it is infellicitous to say the
sentence without direct observation. For instance, it is infellicitous
to say 3b by getting an evidence from hearsay. Then what is the
difference between 3a and 3b? `-te-' in 3b denotes EAT that is prior to
ST. And that EAT is the vantage point from which the evidential
requirement of the assertion should be met. In contrasts 3a has no that
kind of constraints. This can be shown by this example.

Context: storm-a - Amy is watching through the window that it stroms
dangerously outside. After some time, in her house she came across with
her brother who was preparing to go outside. He presumably was in his
basement playing games all day long. She says to her brother, - 3a. 지금
밖에 폭풍쳐. It is storming outside now. - 3b. 지금 밖에 폭풍치더라.
{[}The speaker observed that{]} It is storming outside now.

Context: storm-b - Amy is watching through the window that it stroms
daingously outside. While watching, she is telling her sister who did
not watched the scene (nor heard the sound of storm) yet, - 3a. 지금
밖에 폭풍쳐. It is storming outside now. - 3b. \#지금 밖에 폭풍치더라.
{[}The speaker observed that{]} It is storming outside now.

There is no difference of level of evidence required for Amy to speak 3a
and 3b in both cases. In both cases, the requirement of direct evidence,
or something on a par is required to assert either 3a or 3b. (It is
infelicitous for Amy to say either 3a or 3b without strong evidence.)
However, in the second context, 3b is infellicitous because of -te- in
3b, it has a presupposition that EAT is prior to ST. So, she can't speak
3b, because she is watching the storm now. This clearly shows what -te-
contributes in assertions. It does not create an additional evidential
requirement that wasn't there. But, in -te- sentences, it has a
presupposition that there is a salient time (EAT) before speech time
from which the speaker meet this evidential requirement. 3b in the
second context violated this presupposition. So it is infellicitous.

Ingredients

{[}{[}Supplementary Material for Semantics Term Paper\#Thesis (3)
Evidentiality without ``evidentials''{]}{]} {[}{[}Supplementary Material
for Semantics Term Paper\#Thesis (3) -te- sentences as expressing an
assertion made from a perspective (EAT).{]}{]}

\hypertarget{comparisons-with-previous-accounts}{%
\section{4. Comparisons with previous
accounts}\label{comparisons-with-previous-accounts}}

\hypertarget{cross-linguistic-implications}{%
\section{5. Cross-linguistic
implications}\label{cross-linguistic-implications}}

\hypertarget{conclusion}{%
\section{6. Conclusion}\label{conclusion}}

\hypertarget{references}{%
\section{References}\label{references}}

{[}{[}Korean Evidential -te- Inference from Direct Evidence{]}{]}
\#\#\#\#\#\# Ingredients -

\begin{Shaded}
\begin{Highlighting}[]
\NormalTok{TABLE WITHOUT ID C}
\NormalTok{FROM "ideaBlocks/Semantics Proseminar Term Paper/MVP{-}Evidentiality as Perspectiveness{-} Analysis of Korean morpheme \textquotesingle{}{-}te{-}\textquotesingle{}"}
\NormalTok{WHERE C != null}
\NormalTok{FLATTEN C}
\end{Highlighting}
\end{Shaded}

\begin{Shaded}
\begin{Highlighting}[]
\NormalTok{TABLE Data }
\NormalTok{FROM "ideaBlocks/Semantics Proseminar Term Paper"}
\NormalTok{WHERE Data != null}
\NormalTok{FLATTEN Data}
\end{Highlighting}
\end{Shaded}
